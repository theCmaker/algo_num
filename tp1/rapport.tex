\documentclass{report}
\usepackage[utf8]{inputenc} %encodage entrée
\usepackage{endnotes} %notes de fin
\usepackage{graphicx} %images
\usepackage[usenames,dvipsnames]{color} %couleurs
\usepackage{listings} %mise en forme de code source
\usepackage[frenchb]{babel} %langue
\usepackage{lipsum} %génération de lipsum
\usepackage{verbatim} %code source
\usepackage{moreverb} %amélioration du package verbatim
\usepackage{titlesec} %formatage des chapitres
\titleformat{\chapter}[hang]{\bf\huge}{\thechapter}{2pc}{}
\usepackage[a4paper]{geometry} %mise en page
\geometry{hscale=0.8,vscale=0.8,centering}
%\lstinputlisting[language=Python, firstline=37, lastline=45]{source_filename.py}
\title{Algorithmes numériques -- Rapport}
\author{Axel Delsol, Pierre-Loup Pissavy}
\date{Novembre 2013}
\lstset{literate=
   {á}{{\'a}}1 {é}{{\'e}}1 {í}{{\'i}}1 {ó}{{\'o}}1 {ú}{{\'u}}1
   {Á}{{\'A}}1 {É}{{\'E}}1 {Í}{{\'I}}1 {Ó}{{\'O}}1 {Ú}{{\'U}}1
   {à}{{\`a}}1 {è}{{\'e}}1 {ì}{{\`i}}1 {ò}{{\`o}}1 {ò}{{\`u}}1
   {À}{{\`A}}1 {È}{{\'E}}1 {Ì}{{\`I}}1 {Ò}{{\`O}}1 {Ò}{{\`U}}1
   {ä}{{\"a}}1 {ë}{{\"e}}1 {ï}{{\"i}}1 {ö}{{\"o}}1 {ü}{{\"u}}1
   {Ä}{{\"A}}1 {Ë}{{\"E}}1 {Ï}{{\"I}}1 {Ö}{{\"O}}1 {Ü}{{\"U}}1
   {â}{{\^a}}1 {ê}{{\^e}}1 {î}{{\^i}}1 {ô}{{\^o}}1 {û}{{\^u}}1
   {Â}{{\^A}}1 {Ê}{{\^E}}1 {Î}{{\^I}}1 {Ô}{{\^O}}1 {Û}{{\^U}}1
   {œ}{{\oe}}1 {Œ}{{\OE}}1 {æ}{{\ae}}1 {Æ}{{\AE}}1 {ß}{{\ss}}1
   {ç}{{\c c}}1 {Ç}{{\c C}}1 {ø}{{\o}}1 {å}{{\r a}}1 {Å}{{\r A}}1
   {€}{{\EUR}}1 {£}{{\pounds}}1
}
\lstdefinestyle{customc}{
   belowcaptionskip=1\baselineskip,
   breaklines=true,
   frame=L,
   xleftmargin=\parindent,
   language=C,
   showstringspaces=false,
   basicstyle=\footnotesize\ttfamily,
   keywordstyle=\bfseries\color{ForestGreen},
   commentstyle=\itshape\color{Plum},
   identifierstyle=\color{NavyBlue},
   stringstyle=\color{Orange},
   numbers=left,
}
\begin{document}
  \maketitle
  \tableofcontents

  \chapter*{Introduction}
    \lipsum[1-5]
  \chapter{Structure globale du programme}
    Nous avons choisi de générer un programme principal avec menus, présenté sous la forme suivante :
    \begin{lstlisting}[linewidth=14cm,frame=single,breaklines=true, breakatwhitespace=true]
MENU PRINCIPAL : RESOLUTION D'EQUATIONS LINEAIRES

1. Résolution par Gauss
3. Résolution par Cholesky
2. Résolution par Jacobi
4. Résolution par Gauss-Seidel
5. Résolution par Surrelaxation
6. Génération de matrices carrées
7. Jeux de test
0. Quitter
Votre choix : 
	\end{lstlisting}
	Tous les fonctions de résolution font appel à des fonctions adaptées aux matrices, écrites dans le fichier \verb"matrices.c".
    
    Chaque méthode de résolution reçoit les matrices générées auparavant (juste après le choix de la méthode dans le menu) en arguments.
    
  \chapter{Méthodes directes}
    \section{Méthode de Gauss}
      La méthode de Gauss permet de calculer une solution exacte en un nombre fini d'étapes.
      \newline
      On cherche la matrice $N$ triangulaire supérieure telle que $A = M \cdot N$ avec $M$ la matrice identité.
      \newline
      Critère d'application de l'algorithme :
      \begin{itemize}
        \item{Les éléments diagonaux ne peuvent être nuls,}
        \item{Le déterminant ne doit pas être nul.}
      \end{itemize}
        \lstset{language=C,showstringspaces=false}
      \subsection{Programme}
        \lstinputlisting[style=customc]{gauss.c}
      \subsection{Jeux de tests}
    \newpage
    \section{Méthode de Cholesky}
      \subsection{Programme}
        \lstinputlisting[style=customc]{cholesky.c}
      \subsection{Jeux de tests}
      
    \section{Comparaison}
  \chapter*{Conclusion}
\end{document}